\documentclass{article}
\usepackage[a4paper, hmargin={2.5cm, 2.5cm}, vmargin={2.5cm, 2.5cm}]{geometry}
\usepackage[utf8]{inputenc}
\usepackage[english]{babel}
\usepackage{amsmath,amssymb,graphicx}
\usepackage{cleveref}

\usepackage{listings}
\usepackage{color}

\lstnewenvironment{python}[1][]{
  \lstset{
    language=python,
    breaklines=true,
    tabsize=4,
    basicstyle=\ttfamily\small,
    otherkeywords={1, 2, 3, 4, 5, 6, 7, 8 ,9 , 0, -, =, +, [, ], (, ), \{, \}, :, *, !},
    keywordstyle=\color{blue},
    stringstyle=\color{red},
    showstringspaces=false,
    emph={class, pass, in, for, while, if, is, elif, else, not, and, or, OR
    def, print, exec, break, continue, return},
    emphstyle=\color{black}\bfseries,
    emph={[2]True, False, None, self},
    emphstyle=[2]\color{key},
    emph={[3]from, import, as},
    emphstyle=[3]\color{blue},
    morecomment=[s]{"""}{"""},
    commentstyle=\color{gray}\slshape,
    rulesepcolor=\color{blue},#1
  }
}{}

\title{Insert Assignment Title Here\\02807 Computational Tools for Big Data}
\author{Anonymous authors}
\date{Insert hand in date here}

\begin{document}

\maketitle

\section{Exercise 1.1}

The following pipeline:
\begin{itemize}
\item[1] Deletes all punctuation, commas and quotes from file
\item[2] Translates whitespace to newline
\item[3] Sorts it
\item[4] Counts occurrence of each word
\item[5] Sorts it numerically in reverse (largest number first)
\item[6] Prints the top 10 lines
\end{itemize}

\begin{verbatim}
tr -d ",.'" < test | tr ' ' '\n' | sort | uniq -c | sort -n -r | head -n 10
\end{verbatim}

\section{Exercise 1.2}

The following unix script deletes all lines that contains a number with 5 or more digits

\begin{verbatim}
sed "/[0-9]\{5,\}/d" < test2
\end{verbatim}


\section{Exercise 1.3}

The following pipeline:
\begin{itemize}
\item[1] Translates all tabs into spaces in the shakespeare.txt file
\item[2] Removes all characters satisfying \texttt{[\ $\hat{}$\ a-zA-Z ]}
\item[3] Translates all spaces to newlines
\item[4] Translates upper case to lower case
\item[5] Sorts the lines
\item[6] Keeps only unique lines
\item[7] Uses dict file as plain string to match on the entire individual lines and print only the lines that don't match anything in dict.
\item[8] counts the lines i.e. the misspelled words.
\end{itemize}


\begin{verbatim}
tr '\t' ' ' < shakespeare.txt | sed 's/[^a-zA-Z ]//g' | tr ' ' '\n' | tr A-Z a-z | 
   sort | uniq | grep -F -x -v -f dict | wc -l
\end{verbatim}


\section{Exercise 1.4}
\section{Exercise 1.5}

\section{Exercise 2.1}
\section{Exercise 2.2}
\section{Exercise 2.3}

\section{Exercise 3.1}
\section{Exercise 3.2}
\section{Exercise 3.3}
\section{Exercise 3.4}
\section{Exercise 3.5}

\section{Exercise 4.1}

\end{document}
